\Acrfull{FaaS} applications could harness the disseminated nature of the Fog and take advantage of Fog’s benefits, such as real-time processing and reduced bandwidth. The \gls{FaaS} programming paradigm allows applications to be divided in independent units called “functions.” However, deciding how to place those units in the Fog is challenging. Fog contains diverse, potentially resource-constrained nodes, geographically spanning from Cloud to the network edges. These nodes must be efficiently shared between the multiple applications that will require to use the Fog.

We introduce “Fog node ownership,” a concept where Fog nodes are owned by different actors that chose to give computing resources in exchange for remuneration. This concept allows for the reach of the Fog to be dynamically extended without central supervision from a unique decision taker, as currently considered in the literature. For the final user, the Fog appears as a single unified \gls{FaaS} platform. We use auctions to incentivize Fog nodes to join and compete for executing functions.


Our auctions let Fog nodes independently put a price on candidate functions to run. It introduces the need of a “Marketplace,” a trusted third party to manage the auctions. Clients wanting to run functions communicate their requirements using \acrfullpl{SLA} that provide guarantees over allocated resources or the network latency. Those contracts are propagated from the Marketplace to a node and relayed to neighbors.

This internship focused on providing a straightforward version of our framework for implementing this placement strategy. We support experimenting on Grid'5000 for publishing a first contribution demonstrating the framework's abilities. A comparison to state-of-the-art auctions-enabled placement strategies in the Fog is planned to take place in the near future.

% Additionally, we list the main problems identified by the literature regarding handling applications in the Fog, as well as frameworks doing so. The lack of a proper framework supporting our needs made us develop our own open initiative (\href{https://github.com/VolodiaPG/fog_application_samples}{github.com/VolodiaPG/fog\_application\_samples}). Our major three considerations were ease of extension, the use of community-backed platforms and a painless integration with Grid'5000. In our solution we use Kubernetes as the orchestrator and OpenFaaS as the \gls{FaaS} platform.