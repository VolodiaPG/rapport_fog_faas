\Gls{FaaS} applications could harness the disseminated nature of the Fog and take advantage of Fog’s benefits, such as real-time processing and reduced bandwidth. The \gls{FaaS} programming paradigm allows applications to be divided in independent units called “functions”. However, deciding how to place those units in the Fog is challenging. Fog contains diverse, potentially resource-constrained nodes geographically spanning from Cloud to the network edges. These nodes must be efficiently shared between the multiple applications that will require to use the Fog.

Much research has concentrated on finding appropriate function placement strategies that provision the necessary resources while taking into account—to name a few—distribution, data locality, and latency. \\
The first portion of this report analyzes the existing literature and identifies open research directions, leading to the second part of this document about the internship and paving the way toward a thesis. We will pay attention to various placement architectures, methods, and metrics as well as make a case for utilizing auction-based placement mechanisms.
We think auctions paired with contracts such as \glspl{SLA} would be useful in considering the needs of an application’s functions and in enabling efficient exploitation of the Fog by correctly provisioning those functions.

That culminated in the creation of a new framework compliant with open research standards such as easily accessible documentation and public availability of the project (\href{https://github.com/VolodiaPG/fog_application_samples}{github.com/VolodiaPG/fog\_application\_samples}). It also employs standard software pieces such as Kubernetes to guarantee simple maintenance and full compatibility with global usages and the research based upon it. This framework will allow us to research the introduced auction-centric approach for positioning functions in the Fog. Auctions are hosted on a single « Marketplace »—trusted by everyone—while fog nodes submit a bid representing what they think is the price the function has to pay to be executed on them. Being malleable, the framework will be adapted to compare simple and state-of-the-art placement strategies against ours.

\thesis{A thesis is planned to keep this work going and explore the previous concepts in depth. It will be centered on the consideration of \gls{FaaS}-specific challenges (cold start, function storage, input data). We will also look at the application lifecycle on the Fog as well as the problems that poses to find a suitable auction mechanism respecting properties such as fairness.}