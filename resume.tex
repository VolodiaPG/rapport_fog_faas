Les applications \gls{FaaS} pourraient exploiter la nature disséminée du \emph{Fog} et profiter des avantages du \emph{Fog} tels que le traitement en temps réel et une utilisation de la bande passante réduite. Le paradigme de programmation \gls{FaaS} permet de diviser les applications en unités indépendantes appelées « fonctions ». Cependant, il est difficile de décider comment placer ces unités dans le \emph{Fog}. Le \emph{Fog} contient divers nœuds de calculs potentiellement limités en ressources, allant géographiquement du \emph{Cloud} à la périphérie du réseau. Ces nœuds doivent être partagés efficacement entre les multiples applications qui devront utiliser le \emph{Fog}.

De nombreuses recherches se sont concentrées sur les stratégies de placement de fonctions qui leur donnent accès aux ressources nécessaires tout en tenant compte, entre autres, de la distribution des nœuds, de la localisation des données et de la latence.

La première partie de ce rapport analyse la littérature existante et identifie les directions de recherche existantes, ce qui mène à la deuxième partie de ce document sur le stage ême et qui pave la voie d'une thèse. Nous prêterons attention à diverses architectures, méthodes et mesures de placement, et nous plaiderons en faveur de l'utilisation de mécanismes de placement basés sur les enchères.\\
Nous pensons que les enchères couplées à des contrats tels que les \glspl{SLA} seraient utiles pour prendre en compte les besoins des fonctions d'une application et pour permettre une exploitation efficace du \emph{Fog} en allouant les resources correctement aux fonctions.

Cela a abouti à la création d'un nouveau \emph{framework} conforme aux principes de la recherche ouverte tels que la documentation accessible et la disponibilité publique du projet. Il utilise également des logiciels standards comme l'emploi de Kubernetes pour garantir une maintenance simple et une compatibilité totale avec les usages courants et la recherche se base dessus. Ce \emph{framework} nous permettra d'étudier notre approche centrée sur les enchères pour le positionnement des fonctions dans le \emph{Fog}. Les enchères sont hébergées sur une « place de marché » unique — à laquelle tout le monde fait confiance — tandis que les nœuds du \emph{Fog} soumettent une offre représentant ce qu'ils pensent être le prix que la fonction doit payer pour être exécutée sur leurs resources. Étant malléable, le \emph{framework} sera adapté pour comparer des stratégies de placement simples et de pointe aux nôtres.

\thesis{Une thèse est prévue pour continuer ce travail et explorer en profondeur les concepts précédents. Elle sera centrée sur la prise en compte des défis spécifiques à \gls{FaaS} (démarrage à froid, stockage des fonctions, données d'entrée). Nous nous pencherons également sur le cycle de vie des applications sur le \emph{Fog} ainsi que sur les problèmes engendrés pour le mécanisme d'enchères qui doit respecter des propriétés telles que l'équité.}

\improvement{Fix the french abstract}